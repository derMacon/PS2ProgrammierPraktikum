\section{Problemanalyse und Realisation}
Hierbei habe ich mich einer Technik bedient in der man versucht sich in eine jeweilige Teilkomponente des Problems hineinzuversetzen und anzugeben fuer welchen Teilbereich diese Komponente verantwortlich ist. Anschliessend ist es etwas einfach sowie uebersichtlicher sich auf die Struktur festzulegen, da man saetmliche Nomen als Klassen ansehen kann (hier einmal \colorbox{Apricot}{orange} dargestellt). Verben spiegeln die benoetigten Methoden wieder (hier \colorbox{SpringGreen}{gruen} dargestellt).

\begin{enumerate}
	\item Benutzer
	\begin{enumerate}
		\item als \colorbox{Apricot}{Benutzer} möchte ich den aktuellen Spielstand in eine Datei mit Auswahl des Dateinamens \colorbox{SpringGreen}{speichern}. Das Logging wird mit \colorbox{SpringGreen}{gespeichert}.
		\item als \colorbox{Apricot}{Benutzer} möchte ich eine Datei mit einem Spielstand  \colorbox{SpringGreen}{auswählen} und \colorbox{SpringGreen}{öffnen} können.
		\item als \colorbox{Apricot}{Benutzer} möchte ich ein \colorbox{Apricot}{Spiel} initialisieren und neu starten.
		\item als \colorbox{Apricot}{Benutzer} möchte ich ein \colorbox{Apricot}{Spiel} \colorbox{SpringGreen}{beenden}mit/ohne zu \colorbox{SpringGreen}{speichern}.
		\item als \colorbox{Apricot}{Benutzer} möchte ich das Laden oder Speichern  \colorbox{SpringGreen}{loggen}.
	\end{enumerate}	
	\item Spieler
	\begin{enumerate}
		\item als \colorbox{Apricot}{Spieler} möchte ich einen \colorbox{Apricot}{Domino} \colorbox{SpringGreen}{auswählen}.
 		\item als \colorbox{Apricot}{Spieler} möchte ich einen \colorbox{Apricot}{Domino} \colorbox{SpringGreen}{drehen}.
		\item als \colorbox{Apricot}{Spieler} möchte ich einen \colorbox{Apricot}{Domino} \colorbox{SpringGreen}{setzen}.
		\item als \colorbox{Apricot}{Spieler} möchte ich das Stadtzentrum \colorbox{SpringGreen}{bewegen}.
		\item als \colorbox{Apricot}{Spieler} möchte ich meine Aktionen \colorbox{SpringGreen}{loggen}.
	\end{enumerate}
	\item Spiel
	\begin{enumerate}
		\item als \colorbox{Apricot}{Spiel} möchte ich die \colorbox{Apricot}{Spielfelder} der Spieler \colorbox{SpringGreen}{visualisieren}.
		\item als \colorbox{Apricot}{Spiel} möchte ich die \colorbox{Apricot}{Auswahlfelder} \colorbox{SpringGreen}{visualisieren}.
		\item als \colorbox{Apricot}{Spiel} möchte ich den aktuellen Spielstand der Spieler \colorbox{SpringGreen}{anzeigen}.
		\item als \colorbox{Apricot}{Spiel} möchte ich den \colorbox{Apricot}{Gewinner} \colorbox{SpringGreen}{anzeigen}
		\item als \colorbox{Apricot}{Spiel} möchte ich den \colorbox{Apricot}{Gewinner} \colorbox{SpringGreen}{loggen}.
	\end{enumerate}
\end{enumerate}

Mit dieser Uebersicht bin ich zu folgender groben Klassenstruktur gelagt: 
\begin{enumerate}
	\item Interfaces: Waren zwar mehr oder weniger durch die Bonusaufgabe vorgegeben, aber gerade zum Testen des Programms bietet sich die Verwendung dieser Interfaces natuerlich an. 
	\begin{enumerate}
		\item GUI2Game
		\item GUIConnector
	\end{enumerate}
	\item Klassen
	\begin{enumerate}
		\item Game
		\item Player
		\item Verschiedene Auspraegungen der abstrakten Spieler-Klasse
		\item District
		\item Bank
		\item Entry
		\item Board
		\item Domino
		\item Pos
		\item Logger
	\end{enumerate}
	\item Aufzaehlungstypen
	\begin{enumerate}
		\item Tiles
		\item SingleTile
		\item DistrictType
	\end{enumerate}
\end{enumerate}