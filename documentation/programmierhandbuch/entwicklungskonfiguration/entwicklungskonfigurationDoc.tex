\section{Entwicklungskonfiguration}
\label{sec:entwicklungskonfiguration}
\begin{table}[h!]
	\begin{tabular}{llr} 
	\toprule
	Softwarekomponenten\\  
	\midrule 
	Art & Name & Version\\ 
	\midrule 
	Betriebssystem & Windows & 10 Professional\\ 
	Compiler & Java development kit & 1.8.0\_131\\
  Entwicklungsumgebung & IntelliJ IDEA & 2018.2.3 (Ultimate Edition)\\
  Textbearbeitung & Texmaker & 5.0.3\\
  Bildbearbeitung & GIMP & 2.8\\
  3d Modellierung & Blender & 2.79b\\
	\bottomrule
	\end{tabular}
	\label{tab:ablaufbedingungen}
\end{table}
Au"serdem wurde ein Tool namens Checkstyle verwendet. Checkstyle gibt dem Programmierer einige 
Richtlinien wie der Code am Ende auszusehen hat. Es wird zum Beispiel eine vollst"andige Javadoc-Dokumentation
verlangt, oder Konstanten statt Zahlen ohne Kontext, verlangt. Hierzu wurden die Vorgaben der "Ubung 
\emph{Algorithmen und Datenstrukturen} verwendet. Die entsprechende XML-Datei findet sich hier xxxxx.
\todo{Referenz auf die Checkstyle Datei einbinden}