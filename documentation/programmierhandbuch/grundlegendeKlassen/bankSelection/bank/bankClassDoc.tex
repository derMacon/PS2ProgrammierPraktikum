\paragraph{Bank}
\label{par:bank}
Die Bank-Klasse dient als Datenstruktur auf deren Basis die Spielteilnehmer Dominos f"ur die aktuelle sowie die n"achste Runde w"ahlen k"onnen. Sie verwaltet haupts"achlich einen Array von Entry-Objekten. 

\subparagraph{Konstruktoren}
Die Klasse verf"ugt "uber drei verschiedene Konstruktoren \lstref{lst:bank_konstruktor}.
\begin{enumerate}
	\item Konstruktor: Wird beim standardm"a"sigen Erstellen einer Bank im Spiel genutzt. Es wird lediglich die Anzahl der Spieler gegeben, sodass die Bank entsprechend viele leere Entry-Pl"atze generieren kann. 
	\item Konstruktor: Wird beim Testen ohne Dateiverarbeitung genutzt. Es wird ein bereits generierter Array aus Entry-Objekten gesetzt. Au"serdem besteht die M"oglichkeit ein beliebiges Random Objekt zu setzen (sehr n"utzlich, um vermehrt die selbe Spielsituation generieren zu k"onnen). 
	\item Konstruktor: Wird zum Testen mit Dateiverarbeitung genutzt. Hierbei wird ein String hereingereicht, welcher derartig verarbeitet wird, dass am Ende eine Liste an validen Entry-Daten zustande kommt. 
\end{enumerate}
Bei dem String-Input des Konstruktors wird davon ausgegangen, dass die Syntax stimmt. Diese wird bereits in der Converter-Klasse "uberpr"uft. Bei der Konvertierung des Strings in einen Entry-Array wird der String bei jedem Komma geteilt, sodass es anschlie"send m"oglich ist diesen von hinten nach vorne zu durchlaufen und jeden String-Wert jeweils einem Entry-Konstruktor zu "ubergeben. Dieser ist in der Lage, zusammen mit den teilnehmenden Spielern, ein g"ultiges Entry-Objekt zu generieren. 
\begin{lstlisting}[style=CodeHighlighting,float,caption=Bank - Konstruktor,label=lst:bank_konstruktor]
// standard constructor 
public Bank(int playerCnt) {
    this.entries = new Entry[playerCnt];
    this.bankSize = playerCnt;
        this.rand = new Random();
}

// testing constructor - no fileIO
public Bank(Entry[] entries, Random pseudoRandom) {
    this.entries = entries;
    this.rand = pseudoRandom;
    this.bankSize = entries.length;
}

// testing constructor - with fileIO
public Bank(String preallocation, List<Player> players, Random rand) {
    assert null != preallocation && null != players && null != rand;
    this.bankSize = players.size();
    this.entries = new Entry[this.bankSize];
    if (0 < preallocation.length()) {
        String[] singleEntries = preallocation.split(SEPERATOR_STRING_REPRESENTATION);
        int offset = this.bankSize - singleEntries.length;
        for (int i = singleEntries.length - 1; i >= 0; i--) {
            this.entries[i + offset] = new Entry(singleEntries[i], players);
        }
    }
}
\end{lstlisting}

\subparagraph{Methoden}
Neben zahlreichen Gettern und Settern ist es m"oglich eine Bank aufzul"osen ohne eine neue Instanz anzufordnern, indem man die Methode \emph{clearAllEntries} verwendet, welche lediglich alle Entry-Referenzen auf den Null-Pointer setzt, die Bank-Referenz aber unver"andert l"asst. 

Die wohl wichtigste Methode der Bank-Klasse nennt sich \emph{selectEntry} und erm"oglicht es einem Spieler einen gew"unschten Domino auszuw"ahlen. Hierf"ur wird der Entry an dem gegebenen Bank-Index mit der Spieler-Referenz belegt \lstref{lst:bank_selectEntry}. 

\begin{lstlisting}[style=CodeHighlighting,float,caption=Bank - selectEntry,label=lst:bank_selectEntry]
public void selectEntry(Player player, int bankIdx) {
    assert null != player && isValidBankIdx(bankIdx) && null != this.entries
            && null != this.entries[bankIdx] && isNotSelected(bankIdx);
    this.entries[bankIdx].selectEntry(player);
}
\end{lstlisting}