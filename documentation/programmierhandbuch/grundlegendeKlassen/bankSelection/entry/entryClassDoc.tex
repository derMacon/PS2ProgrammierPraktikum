\paragraph{Entry}
\label{par:entry}
Diese Klasse dient als Datenstruktur f"ur die einzelnen Bankelemente. Man h"atte hier auch eine Map oder ein Pair verwenden k"onnen, mir waren diese Datentypen zum Zeitpunkt des Entwurfes allerdings noch nicht bekannt da ich erst im laufenden Semester die Veranstaltung \emph{Algorithmen und Datenstrukturen} besucht habe. 

\subparagraph{Konstruktor}
Diese Klasse besitzt zwei Felder zum Speichern eines Dominos und dem Spieler der diesen Domino verwendet. Im ersten Konstruktor werden diese ohne weitere Aktionen gesetzt. Im zweiten wird eine Zeichenfolge und die Liste der teilnehmenden Spieler hereingereicht. Es wird hierbei davon ausgegangen, dass der String bereits beim initialen Einlesen derartig "uberpr"uft wurde, dass die Syntax den Richtlinien entspricht. Nun wird der eingegebene String an der Stelle geteilt, an der sich das Leerzeichen befindet. Der vordere Teil repr"asentiert den Spieler und wird daher zu einem Integer geparsed (falls dies m"oglich sein sollte). Anschlie"send wird der Domino gebildet. Neben zahlreichen Gettern und Settern besitzt die Klasse noch eine \emph{toString}- sowie eine \emph{copy}- und \emph{equals}-Methode. 
