\paragraph{Entry}
\label{par:entry}

\subparagraph{Einleitung}
Diese Klasse dient als Datenstruktur f"ur die einzelnen Bankelemente. Sie verh"alt sich "ahnlich zu einer herk"ommlichen Map. Allerdings bietet sie eine spezialisierte \emph{toString}-Methode um den Richtlinien zu gen"ugen, sowie eine \emph{"uberarbeitete}-equals Modthode um das Comparable interface zu implementieren. Desweiteren habe ich die Veranstaltung \emph{Algorithmen und Datenstrukturen} erst im laufendem Semester besucht und kannte diese Datenstruktur zum Zeitpunkt des Entwurfs noch nicht. 

\subparagraph{Konstruktor}
Diese Klasse besitzt zwei Felder zum speichern eines Dominos und dem Spieler der diesen Domino verwendet. Im ersten Konstruktor werden diese ohne weitere Aktionen gesetzt. Im zweiten, wird eine Zeichenfolge und die Liste der teilnehmenden Spieler hereingereicht. Es wird hierbei davon ausgegangen das der String bereits beim initialen Einlesen derartig "uberpr"uft wurde, dass die Syntax den Richtilinien entspricht. 
\todo{Referenz zu Syntax-Richtlinien einbetten.}
Nun wird der eingegebene String an der Stelle geteilt an der sich das Leerzeichen befindet. Der vordere Teil repr"asentiert den Spieler. Dieser Teil wird mit dem Zeichen f"ur einen nicht ausgew"ahlten Slot verglichen und anschlie"send mit dem 
\todo{noch nicht fertig}

\subparagraph{•}
