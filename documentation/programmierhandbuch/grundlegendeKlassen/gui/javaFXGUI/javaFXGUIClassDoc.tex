\paragraph{JavaFXGUIClass}
\label{par:javaFXGUI}

\subparagraph{Einleitung}
Diese Klasse implementiert das Interface GUIConnector und somit die Logik, welche f"ur die Darstellung einer Partie ben"otigt wird. Sie h"alt ("ahnlich wie die Game-Klasse selbst) s"amtliche Teilkomponenten in einer Datenstruktur. Die hierbei verwendeten Datenstrukturen beziehen sich allerdings direkt auf die Darstellungsebene. B"anke werden hier beispielsweise nicht vom Datentyp Bank beschrieben, sondern besitzen den Typ \emph{ImageView[][]}. Zu gro"sen Teilen war auch diese Klasse bereits aus der Bonusaufgabe vorgegeben und wurde entsprechend "ubernommen. 

\subparagraph{Methoden}
S"amtliche Daten werden hier in Form von Bildern gespeichert. Ein Aufruf von zum Beispiel \emph{setToBank} tut hierbei nichts anderes, als die Bilder der einzelnen \emph{Tiles} des gegebenen Dominos auf der gew"unschten Bank zu setzen. "Ahnlich arbeitet die Methode \emph{showCellOnGrid}. Um derartige Methoden nutzen zu k"onnen muss die Klasse in der Lage sein, diese zu \glqq "ubersetzen \grqq . Hierzu wird im Konstruktor ein entstprechendes Array an Bildern initialisiert, um nun das gew"unschte Bild zu erhalten reicht es den Ordinal-Wert des jeweiligen \emph{Tiles} zu ermitteln und als index in das Array zu reichen. 