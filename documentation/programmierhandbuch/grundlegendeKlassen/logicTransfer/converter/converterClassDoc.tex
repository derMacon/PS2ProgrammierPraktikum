\paragraph{Converter}
\label{par:converter}

\subparagraph{Einleitung}
Diese Klasse besch"afftigt sich mit dem Generieren s"amtlicher Teilkomponenten der Game-Klasse aus einem gegebenen String. Neben dem Erstellen der Objekte ist sie au"serdem f"ur die "Uberpr"ufung der Syntax verantwortlich, das Laden der Dateien wurde hier allerdings explizit abgegrenzt und findet in der \emph{Loader}-Klasse statt. Um einen String zu interpretieren wird die Methode \emph{readStr} verwendet \lstref{lst:converter_readStr}. 

\begin{lstlisting}[style=CodeHighlighting,float,caption=Converter - readStr,label=lst:converter_readStr]
public String readStr(GUIConnector gui, String input) {
    try {
        if (input == null || input.length() == 0) {
            throw new IOException(UNSUCCESSFUL_READ_MESSAGE);
        }
        // Tag syntax roughly checked -> further analysis further down the line
        if (!input.matches(MATCHES_TAGS)) {
            System.out.println(input);
            throw new WrongTagException();
        }
        String[][] descriptionBlocks = genDescriptiveField(input);
        fillFieldsWithDescriptiveBlocks(descriptionBlocks, gui);
        return SUCCESSFUL_READ_MESSAGE;
    } catch (Exception e) {
        return e.getMessage();
    }
}
\end{lstlisting}

\begin{enumerate}
	\item Schritt: Es wird eine grobe Struktur geschaffen um einfach s"amtliche Teilbereiche analysieren zu k"onnen. Hierbei wird ein zweidimensionales Array gebildet, welches in der ersten Dimension jeweils den Tag-Bezeichner des jeweiligen Objektes (<Spielfeld...>, <B"anke> oder <Stapel>) und in der zweiten die wirklichen Daten enth"alt. 
	\item Schritt: Die einzelnen Felder des Arrays werden auf ihre syntaktische Richtigkeit "uberpr"uft.
	\item Schritt: Die Strings in den Array-Feldern werden zu Objekten konvertiert. 
\end{enumerate}

\subparagraph{Rohdaten aufteilen}
\label{spar:converter_ablauf}
Bevor es zur wirklichen Aufteilung der Rohdaten kommt wird initial schon mal "uberpr"uft ob die Bezeichener stimmen oder nicht. Dies wird "uber folgenden reg"ul"aren Ausdurck getan: (MATCHES\_TAGS)
\\ \verb|"(<Spielfeld[^>]*>\n(?s)[^<>]*)*<Bänke>\n(?s)[^<>]*<Beutel>\n[^<>]*"|
\\ Hierbei wird abgepr"uft ob die Bezeichner Namen stimmen und mit spitzen Klammern eingeleitet sowie beendet werden. Nach dieser ersten Pr"ufung wird das Array mit den Rohdaten erstellt. Hierzu wird die Methode genDescripitveField aufgerufen \lstref{lst:converter_genDescriptiveField}. Hier werden im ersten Schritt die Bl"ocke in die einzelnen Komponenten aufgebrochen, es wird n"amlich bei jeder "offnenden spitzen Klammer geteilt. Im zweiten Schritt werden diese noch weiter verfeinert in dem bei jeder schlie"senden spitzen Klammer geteilt wird. Man erlangt also grob die Darstellung aus Tabelle \ref{tab:bspRochdaten}. Anschlie"send werden die Bezeichner aus den ersten Feldern herausgel"ost (es werden also zum Beispiel alle spitzen Klammern verworfen).

Im n"achsten Schritt werden die Daten "uber die Funktion \emph{fillFieldsWithDescriptiveBlocks} interpretiert \lstref{lst:converter_fillFieldsWithDescriptiveBlocks}. Hier werden die im vorherigen Schritt erzeugten Felder durchlaufen, es wird jeweils der das Feld mit dem Bezeichner zugeordnet. F"ur jeden unterschiedlichen Bezeichner gibt es eine eigene Methode zum interpretieren der Daten. Wenn man zum Beispiel mit i == 0 auf ein Feld mit einem Spielfeld-Bezeichner im erst Feld des zweidimensionalen Arrays zugreift wird im Case-Verteiler der BOARD\_IDENTIFIER greifen und es wird zuerst einmal die Spielfeld-Syntax "uberpr"uft. Hierbei spielen Dinge wie Leerzeichen oder die Dimensionen des Bretts eine Rolle, aber auch invalide Zellen werden abgefangen. Da in meinem Modell die Spieler ein Feld mit dem Brett besitzen, habe ich den Besitzer des Spielfeldes gleich mit initialisiert, da so die Liste der Distrikte gleich mit aufgebaut werden. Hierzu wird die Methode \emph{convertStrToPlayerWithDefaultOccupancy} aufgerufen. Diese ist eine Hilfsmethode und macht nichts anderes als anhand des "ubergebenen Indices festzustellen ob es sich um einen Bot oder um den menschlichen Spieler handelt. Um den Spieler letztendlich zu initialisieren wird die statische Factory-Methode des PlayerType Enums aufgerufen. Wieso hier eine Factory benutzt wird, wird im Abschnitt \ref{par:playerType} auf S. \pageref{par:playerType} gelk"art. 

\todo{fillDescriptiveBlocks - Bank als naechstes dran}

\begin{table}
\centering
\begin{tabular}{ll}
\toprule
Beispiel\\
\midrule
Bezeichner & Daten\\
\midrule
Spielfeld 1 & -- -- -- -- --, etc.\\
Spielfeld 2 & -- -- -- -- --, etc.\\
Spielfeld 3 & -- -- -- -- --, etc.\\
Spielfeld 4 & -- -- -- -- --, etc.\\
B"anke & 3 A1H0,- A1H0,- A1H0,1 P0S1\\
Stapel & P0P0,H0H0,P0S0,H0A0\\
\bottomrule
\end{tabular}
\caption{1. Schritt: Rohdaten grob aufgeteilt}
\label{tab:bspRochdaten}
\end{table}
 
\begin{lstlisting}[style=CodeHighlighting,float,caption=Converter - genDescriptiveField,label=lst:converter_genDescriptiveField]
public String[][] genDescriptiveField(String input) throws WrongTagException {
    List<String> blocks = new LinkedList<>();
    // overall sections (board/banks/stack) are seperated
    for (String currBlock : input.split("<")) {
        blocks.add(currBlock);
    }
    blocks.remove(0); // First element may be empty because of split()

    // Data seperated from Identifier
    String[][] output = new String[blocks.size()][2];
    for (int i = 0; i < blocks.size(); i++) {
        output[i][DESCRIPTION_IDX] = genTag(blocks.get(i));
        output[i][DATA_IDX] = genData(blocks.get(i));
    }

    return output;
}
\end{lstlisting}

\begin{lstlisting}[style=CodeHighlighting,float,caption=Converter - fillFieldsWithDescriptiveBlocks,label=lst:converter_fillFieldsWithDescriptiveBlocks]
public void fillFieldsWithDescriptiveBlocks(String[][] descriptionBlocks, 
			GUIConnector gui)
        throws WrongTagException, WrongBoardSyntaxException, WrongBankSyntaxException,
        WrongStackSyntaxException {
    // TODO delete before final commit
    int[] dimensions = new int[]{NOT_INITIALIZED, NOT_INITIALIZED};
    for (int i = 0; i < descriptionBlocks.length; i++) {
        switch (descriptionBlocks[i][DESCRIPTION_IDX]) {
            case BOARD_IDENTIFIER:
                dimensions = checkBoardSyntax(dimensions, descriptionBlocks[i][DATA_IDX]);
                this.players.add(i, convertStrToPlayerWithDefaultOccupancy(
                        descriptionBlocks[i][DATA_IDX], i, gui));
                break;
            case BANK_IDENTIFIER:
                checkBankSyntax(descriptionBlocks[i][DATA_IDX], this.players.size());
                Bank[] banks = convertStrToBanks(descriptionBlocks[i][DATA_IDX]);
                this.currentBank = banks[Game.CURRENT_BANK_IDX];
                this.nextBank = banks[Game.NEXT_BANK_IDX];
                this.currBankPos = 4 - descriptionBlocks[Game.CURRENT_BANK_IDX].length;
                break;
            case STACK_IDENTIFIER:
                checkStackSyntax(descriptionBlocks[i][DATA_IDX]);
                this.stack = convertStrToStack(descriptionBlocks[i][DATA_IDX]);
                break;
            default:
                throw new WrongTagException(String.format(
                		WrongTagException.DEFAULT_MESSAGE, 
                		descriptionBlocks[i][DESCRIPTION_IDX]));
        }
    }
}
\end{lstlisting}
