\section{Beschreibung grundlegender Klassen}
\sectionmark{Grundlegende Klassen}
Alle Klassen wurden in folgende Packages unterteilt. Im folgenden werden diese vom Gesamtprogram isoliert betrachtet. Eine globale "Ubersicht "uber s"amtliche Zusammenh"ange findet sich erst im Kapitel ...
\todo{Kapitelref einfuegen}
\begin{itemize}
	\item \ref{ss:token} \nameref{ss:token}
	\begin{itemize}
		\item \nameref{par:pos}
		\item \nameref{par:districtType}
		\item \nameref{par:tiles}
		\item \nameref{par:singleTile}
		\item \nameref{par:domino}
	\end{itemize}
	\item \ref{ss:dataPreservation} \nameref{ss:dataPreservation}
	\begin{itemize}
		\item \nameref{par:loader}
		\item \nameref{par:logger}
	\end{itemize}
\end{itemize}
\import{programmierhandbuch/grundlegendeKlassen/token/}{tokenPackageDoc.tex}
\newpage
\import{programmierhandbuch/grundlegendeKlassen/dataPreservation/}{dataPreservationPackageDoc.tex}
\newpage
\import{programmierhandbuch/grundlegendeKlassen/playerState/}{playerStatePackageMain.tex}
\newpage
\import{programmierhandbuch/grundlegendeKlassen/bankSelection/}{bankSelectionPackageMain.tex}
\newpage
\import{programmierhandbuch/grundlegendeKlassen/differentPlayerTypes/}{differentPlayerTypesPackageMain.tex}
\newpage
\import{programmierhandbuch/grundlegendeKlassen/logicTransfer/}{logicTransferPackageMain.tex}
