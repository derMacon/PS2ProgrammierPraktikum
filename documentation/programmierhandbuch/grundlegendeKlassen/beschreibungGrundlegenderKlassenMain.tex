\section{Beschreibung grundlegender Klassen}
\sectionmark{Grundlegende Klassen}
Alle Klassen wurden in folgende Packages unterteilt. Im folgenden werden diese vom Gesamtprogram isoliert betrachtet. Eine globale "Ubersicht "uber s"amtliche Zusammenh"ange findet sich erst im Kapitel \ref{sec:programmorganisationsplan} \nameref{sec:programmorganisationsplan}, auf Seite \pageref{sec:programmorganisationsplan}. 


{\large gui-Package}


{\large Logic-Package}
\begin{multicols}{2}
\begin{itemize}

	\item \ref{ss:token} \nameref{ss:token}
	\begin{itemize}
		\item \nameref{par:pos}
		\item \nameref{par:districtType}
		\item \nameref{par:tiles}
		\item \nameref{par:singleTile}
		\item \nameref{par:domino}
	\end{itemize}
	\item \ref{ss:dataPreservation} \nameref{ss:dataPreservation}
	\begin{itemize}
		\item \nameref{par:loader}
		\item \nameref{par:logger}
	\end{itemize}
	\item \ref{ss:bankSelection} \nameref{ss:bankSelection}
	\begin{itemize}
		\item \nameref{par:bank}
		\item \nameref{par:choose}
		\item \nameref{par:entry}
	\end{itemize}
	\begin{minipage}{\linewidth}
	\item \ref{ss:playerState} \nameref{ss:playerState}
	\begin{itemize}
		\item \nameref{par:district}
		\item \nameref{par:board}
		\item \nameref{par:botbehavior}
		\item \nameref{par:result}		
		\item \nameref{par:resultRanking}	
	\end{itemize}
	\end{minipage}
	\item \ref{ss:differentPlayerTypes} \nameref{ss:differentPlayerTypes}
	\begin{itemize}
		\item \nameref{par:defaultAIPlayer}
		\item \nameref{par:humanPlayer}
		\item \nameref{par:playerType}
	\end{itemize}
	\item \ref{ss:logicTransfer} \nameref{ss:logicTransfer}
	\begin{itemize}
		\item \nameref{par:exception}
		\item \nameref{par:converter}
		\item \nameref{par:gui2Game}		
		\item \nameref{par:guiConnector}
		\item \nameref{par:possibleField}		
		\item \nameref{par:game}
	\end{itemize}
\end{itemize}
\end{multicols}

\newpage
\import{programmierhandbuch/grundlegendeKlassen/gui/}{guiPackageMain.tex}
\newpage

\import{programmierhandbuch/grundlegendeKlassen/token/}{tokenPackageDoc.tex}
\newpage
\import{programmierhandbuch/grundlegendeKlassen/dataPreservation/}{dataPreservationPackageDoc.tex}
\newpage
\import{programmierhandbuch/grundlegendeKlassen/bankSelection/}{bankSelectionPackageMain.tex}
\newpage
\import{programmierhandbuch/grundlegendeKlassen/playerState/}{playerStatePackageMain.tex}
\newpage
\import{programmierhandbuch/grundlegendeKlassen/differentPlayerTypes/}{differentPlayerTypesPackageMain.tex}
\newpage
\import{programmierhandbuch/grundlegendeKlassen/logicTransfer/}{logicTransferPackageMain.tex}
