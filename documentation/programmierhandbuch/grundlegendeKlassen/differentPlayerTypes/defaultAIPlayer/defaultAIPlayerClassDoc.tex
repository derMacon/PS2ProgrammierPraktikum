\paragraph{DefaultAIPlayer}
\label{par:defaultAIPlayer}

\subparagraph{Einleitung}
Diese Klasse implementiert die in der Aufgabenstellung vorgegebene k"unstliche Intelligenz. Sie erbt von der Player-Klasse um auf s"amtliche Grundfunktionen und Attribute die jeder Spieler haben sollte, zugreifen zu k"onnen. Au"serdem implementiert sie das Interface \emph{Botbehavior} um die KI spezifischen Methoden zu implementieren. 

\subparagraph{Methoden}
Der Spieler ist in der Lage zwischen dem initialen Selektieren auf der Bank der aktuellen Runde und dem auf der Bank der n"achsten Runde zu unterscheiden. Hierzu werden zwei unterschiedliche Methoden verwendet. 

\emph{doInitialSelect} selektiert auf der \emph{aktuellen} Bank \lstref{lst:defaultAIPlayer_doInitialSelect}. Hierbei gilt es zu beachten, dass der Bot direkt nach dem Selektieren bereits den Domino auf seinem Board positioniert. Dies ist erforderlich um diesen beim Selektieren des n"achsten Dominos bei der Berechnung der Position mit einzubeziehen. 

\begin{lstlisting}[float,style=CodeHighlighting,caption=DefaultAIPlayer - doInitialSelect,label=lst:defaultAIPlayer_doInitialSelect]
@Override
public Bank doInitialSelect(Bank currBank, int bankOrd) {
    Bank output = selectFromBank(currBank, bankOrd, true);
    Domino playerSelectedDomino = currBank.getPlayerSelectedDomino(this);
    // update board -> has to be done to prevent the bot from laying the
    // second draft directly on the first domino
    this.board.lay(playerSelectedDomino);
    return output;
}
\end{lstlisting}

Um einen Domino auf einer Bank zu selektieren wird die Methode \emph{selectFromBank} verwendet \lstref{lst:defaultAIPlayer_selectFromBank}. Diese sucht f"ur jeden Domino der Bank die am besten geeignete Position und schreibt die am besten geeignete Position in ein Choose-Objekt. Dies geschieht f"ur jeden Domino, sodass eine Liste geformt wird. Diese Liste wird darauf untersucht welches Choose-Objekt insgesamt am besten geeignet ist. Dieses Choose-Objekt wird anschlie"send weitergereicht um das Selektieren auf der normalen Bank zu ver"andern. Im Anschluss wird die interne Board-Repr"asentation mit dem Domino belegt und es wird per Flag entschieden ob diese "Anderung auch auf der Gui dargestellt werden soll (Testdurchlauf oder richtiger Spieldurchlauf). Nachdem alle n"otigen Z"uge veranlasst wurden, wird alles in der Logdatei festgehalten. Die Methode \emph{bestOverallChoose} ruft eine Schleife auf mit welcher alle Dominos der Bank durchgegangen werden und an die Methode \emph{genBestChoose} weitergereicht werden \lstref{lst:defaultAIPlayer_genBestChoose}. Diese Methode setzt die Positition des Dominos einzeln auf alle Felder des Bretts und schaut ob der Domino passt. Dies ist so nur m"oglich, da vorher ein Deep-Copy angefertig wurde um den Domino auf der Bank nicht zu ver"andern. Falls dies der Fall ist wird ein neues Choose-Objekt erstellt. Falls dieses effizienter sein sollte als das aktuelle maximale Choose-Objekt, "uberschreibt dieses das alte \lstref{lst:defaultAIPlayer_mostEfficient}. 

\begin{lstlisting}[float,style=CodeHighlighting,caption=DefaultAIPlayer - selectFromBank,label=lst:defaultAIPlayer_selectFromBank]
@Override
public Bank selectFromBank(Bank bank, int ordBank, boolean displayOnGui) {
    if (null == bank || bank.isEmpty()) {
        return bank;
    }
    Bank bankCopy = bank.copy();
    List<Choose> bestChoosesForEachPossibleBankSlot = bestChooseForEachDom(bankCopy);
    Choose overallBestChoose = bestOverallChoose(bankCopy, 
    	bestChoosesForEachPossibleBankSlot);
    doSelect(bank, overallBestChoose);
    updateBoard(ordBank, displayOnGui, overallBestChoose);

    // log selection
    String roundIdentifier = ordBank == 0 ? "current" : "next";
    Logger.getInstance().printAndSafe("\n" + String.format(
    		Logger.SELECTION_LOGGER_FORMAT,
            getName(), overallBestChoose.getDomWithPosAndRot().toString(),
            overallBestChoose.getIdxOnBank(), roundIdentifier));

    // return the bank, although bank reference is modified internally
    // (just to make sure it is evident, pos and rot modified)
    return bank;
}
\end{lstlisting}

\begin{lstlisting}[float,style=CodeHighlighting,caption=DefaultAIPlayer - genBestChoose,label=lst:defaultAIPlayer_genBestChoose]
private Choose genBestChoose(Domino domino, int bankSlotIndex) {
    Choose currChoose;
    Choose maxChoose = null;
    for (int y = 0; y < this.board.getSizeY(); y++) {
        for (int x = 0; x < this.board.getSizeX(); x++) {
            domino.setPos(new Pos(x, y));
            for (int i = 0; i < Board.Direction.values().length; i++) {
                if (this.board.fits(domino)) {
                    currChoose = genChoose(domino, bankSlotIndex);
                    maxChoose = mostEfficient(maxChoose, currChoose);
                }
                domino.incRot();
            }
        }
    }
    return null == maxChoose ? genChoose(domino.setPos(new Pos(0, 0)), 
    				bankSlotIndex) : maxChoose;
}
\end{lstlisting}

\begin{lstlisting}[float,style=CodeHighlighting,caption=DefaultAIPlayer - mostEfficient,label=lst:defaultAIPlayer_mostEfficient]
private Choose mostEfficient(Choose fstChoose, Choose sndChoose) {
    Choose res;
    if (null == fstChoose) {
        res = sndChoose;
    } else if (null == sndChoose) {
        res = fstChoose;
    } else if (fstChoose.getPotentialPointsOnBoard() 
    			> sndChoose.getPotentialPointsOnBoard()) {
        res = fstChoose;
    } else if (fstChoose.getPotentialPointsOnBoard() 
    			< sndChoose.getPotentialPointsOnBoard()) {
        res = sndChoose;
    } else {
        // tie
        int fstSingleCellCount = countSingleCells(fstChoose);
        int sndSingleCellCount = countSingleCells(sndChoose);

        if (fstSingleCellCount < sndSingleCellCount) {
            res = fstChoose;
        } else if (fstSingleCellCount > sndSingleCellCount) {
            res = sndChoose;
        } else {
            res = fstChoose.getDomWithPosAndRot().compareTo(
            	sndChoose.getDomWithPosAndRot()) <= 0 ? fstChoose
                    : sndChoose;
        }
    }
    return res;
}

private int countSingleCells(Choose choose) {
    assert null != choose;
    List<District> temp = updateDistricts(this.districts, choose.getDomWithPosAndRot());
    int out = 0;
    for (District currDistrict : temp) {
        if (currDistrict.getTilePositions().size() == 1) {
            out++;
        }
    }
    return out;
}
\end{lstlisting}

\todo{Falls noch Zeit ist den Selektiervorgang genauer Beschreiben}

\todo{Es fehlen doLastTurn und doStandardTurn}