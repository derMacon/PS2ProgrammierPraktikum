\paragraph{player}
\label{par:Player}
\subparagraph{Einleitung}
Diese Klasse legt die Grundfunktionen der einzelnen Spielertypen fest. Sie ist abstrakt da es durch den menschlichen Spieler und den vorgegebenen k"unstlichen Spielertyp mindestens zwei unterschiedliche Spielertypen geben muss. 

\subparagraph{Felder}
Ein Spieler besitzt einen Index im Array der \emph{Game}-Klasse, welcher in der Player-Instanz selbst gespeichert wird. Dieser dient immer wieder als Anhaltspunkt um welchen Spieler es sich im Augenblick handelt. Weiterhin besitzt ein Spieler ein \emph{Board} auf dem dieser verschiedene Dominos setzen kann. Um die erzielten Punkte jederzeit direkt errechnen zu k"onnen, ohne jedes Mal das Board aufs Neue untersuchen zu m"ussen, verwaltet der Spieler ein Liste von Distrikten. Weiterhin ist der Spieler, durch eine Gui-Referenz, in der Lage seine Z"uge direkt auf der Gui darzustellen. 

\subparagraph{Konstruktor}
Es gibt drei verschiedene Auspr"agungen des Konstruktors (siehe Listing \ref{lst:player_konstruktor}, \nameref{lst:player_konstruktor}). Bei dem Konstruktor welcher einen String einliest wird ebenfalls eine Liste an Distrikten gebildet. Im Abschnitt \nameref{spar:aufbauDerDistrikte} wird dies n"aher erleutert.

\begin{lstlisting}[style=CodeHighlighting,float,caption=Player - Konstruktoren,label=lst:player_konstruktor]
// Konstruktor zum starten des Spiels
public Player(GUIConnector gui, int idxInPlayerArray, int boardSizeX, int boardSizeY) {
    this(gui, idxInPlayerArray, new Board(boardSizeX, boardSizeY));
}

// Konstruktor zum Testen ohne Dateiverarbeitung
public Player(GUIConnector gui, int idxInPlayerArray, Board board) {
    this.idxInPlayerArray = idxInPlayerArray;
    this.districts = new LinkedList<>();
    this.gui = gui;
    this.board = board;
}

// Konstruktor zum Testen mit Dateiverarbeitung
public Player(GUIConnector gui, int idxInPlayerArray, String strBoard) {
    this(gui, idxInPlayerArray, new Board(strBoard));
    this.districts = genDistrictsFromBoard(this.board);
}
\end{lstlisting}


\subparagraph{Aufbau der Distrikte}
\label{spar:aufbauDerDistrikte}
In der Methode \emph{genDistrictsFromBoard} werden alle Zeilen eines gegebene Board einzeln untersucht und in eine Liste von Distrikten entsprechend eingeordnet (siehe Listing \ref{lst:player_genDistrictsFromBoard}). Hierzu wird eine Methode namens \emph{addToAppropriateDistrict} aufgerufen (siehe Listing \ref{lst:player_addToAppropriateDistrict}). 

Diese Methode erweitert eine gegebene Liste an Distrikten an der Richtigen Stelle mit einem neuen Element, welches aus dem gegebenen Tile-Objekt und der Position zusammengesetzt wird. Allerdings wird die Liste nur erweitert, wenn das Tile-Objekt ungleich dem der leeren Zelle und dem Stadtzentrum ist. Anschlie"send wird folgendes Schema angewendet:  (siehe Listing \ref{lst:player_addToAppropriateDistrict})
\begin{itemize}
	\item Zeile 4: Es werden alle Distrikte zwischengespeichert, welche an die gegebene Position angrenzen und den selben Distrikt-Typen aufweisen wie das gegebene Tile-Objekt. Falls es keine solcher Distrikte geben sollte, wird eine neuer Distrikt erzeugt und ebenfalls zwischengespeichert. 
	\item Zeile 6: Nun werden alle generierten / gefundenen Distrikte aus der Liste an Distrikten, welche der Player verwaltet, gel"oscht um Duplikate zu vermeiden wenn man sp"ater die Distrikte wieder zusammenf"uhrt.
	\item Zeile 7: Die Distrikte vom Zwischenergebnis wird per Konstruktoraufruf der Distriktklasse zusammengef"uhrt. Das Zusammenf"uhren \emph{mehrerer} Distrikte tritt nur auf wenn das gegebene Tile-Objekt in mehrere Distrikte passt. 
	\item Zeile 8: Da das SingleTile-Objekt und die Position bisher nur benutzt wurden um Vergleiche zu bewerkstelligen, werden diese nun in den zusammengef"uhrten Distrikt eingebunden. 
	\item Zeile 9: Der neu generierte / "uberarbeitete Distrikt wird in die Distrikt-Liste des Spielers "uberf"uhrt. 
\end{itemize}
\begin{lstlisting}[style=CodeHighlighting,float,caption=Player - genDistrictsFromBoard,label=lst:player_genDistrictsFromBoard]
private List<District> genDistrictsFromBoard(Board board) {
    List<District> futureDistrictList = new LinkedList<>();
    SingleTile[][] cells = board.getCells();
    for (int y = 0; y < board.getSizeY(); y++) {
        for (int x = 0; x < board.getSizeX(); x++) {
            futureDistrictList = addToAppropriateDistrict(cells[x][y], 
            	new Pos(x, y), futureDistrictList);
        }
    }
    return futureDistrictList;
}
\end{lstlisting}
\begin{lstlisting}[style=CodeHighlighting,float,caption=Player - addToAppropriateDistrict,label=lst:player_addToAppropriateDistrict]
private List<District> addToAppropriateDistrict(SingleTile tile, Pos pos, 
			List<District> districts) {
    if (SingleTile.EC != tile && SingleTile.CC != tile) {
        List<District> possibleDistricts = findPossibleDistricts(tile, pos, districts);
        districts.removeAll(possibleDistricts); // to avoid duplicates
        District updatedDistrict = new District(possibleDistricts); // merging districts
        updatedDistrict.add(tile, pos); // put new element in merged playdistrict
        districts.add(updatedDistrict);
    }
    return districts;
}
\end{lstlisting}
Die Methode \emph{findPossibleDistricts}, welche in der 4. Zeile verwendet wurde, durchsucht alle Distrikte nach allen Distrikten welche an die gegebene Position angrenzen und den selben Distrikttypen wie das gegebene SingleTile-Objekt besitzen (siehe Listing \ref{lst:player_findPossibleDistricts}). Es wird hierbei auf eine Methode der Distrikt Klasse namens \emph{typeAndPosMatchCurrDistrict} zugegriffen, welche genau das gerade beschriebene Verhalten implementiert. In der Erkl"arung der Distrikt-Klasse wird hierauf n"aher eingegangen (siehe Abschnitt \ref{spar:districtMethoden}, \nameref{spar:districtMethoden}). Falls die Pr"ufung zutreffen sollte, wird die Ausgabeliste mit dem Distrikt gef"ullt und nachdem alle Distrikte durchlaufen wurden zur"uckgegeben. 
\begin{lstlisting}[style=CodeHighlighting,float,caption=Player - findPossibleDistricts,label=lst:player_findPossibleDistricts]
private List<District> findOrCreatePossibleDistricts(SingleTile tile, Pos pos, 
		final List<District> districts) {
    List<District> filteredDistrictList = new LinkedList<>();
    for (District currDistrict : districts) {
        if (currDistrict.typeAndPosMatchCurrDistrict(tile, pos)) {
            filteredDistrictList.add(currDistrict);
        }
    }
    return filteredDistrictList;
}
\end{lstlisting}

Desweiteren bietet die Player Klasse noch eine M"oglichkeit die bestehende Liste an bestehenden Distrikten zu "uberarbeiten ohne die bestehende Distriktliste zu ver"andern. Die Methode \emph{updateDistricts} (siehe Listing \ref{lst:player_updateDistricts}, \nameref{lst:player_updateDistricts}) generiert zuerst ein Kopie des Distrikts (deep copy: neue Referenzen) um anschlie"send einen gegebenen Domino in diese Kopie einzupflegen (siehe Listing \ref{lst:player_addToAppropriateDistrict}, \nameref{lst:player_addToAppropriateDistrict}). Dieser Mechanismus wird vorallem vom k"unstlichen Spieler ben"otigt, da dieser viele Dominos "uberpr"ufen m"ochte und f"ur jeden dieser Dominos die Punkteanzahl berechnet. Hierbei soll die bestehende Distriktliste nicht ver"andert werden, dazu aber in der Beschreibung des \emph{DefaultAIPlayers} mehr. 
\begin{lstlisting}[style=CodeHighlighting,float,caption=Player - updateDistricts,label=lst:player_updateDistricts]
protected List<District> updateDistricts(final List<District> districts, Domino domino) {
    // deep copy of whole district list
    List<District> output = new LinkedList<>();
    for (District currDistrict : districts) {
        output.add(currDistrict);
    }
    // adding domino to slots
    output = addToAppropriateDistrict(domino.getFstVal(), domino.getFstPos(), output);
    output = addToAppropriateDistrict(domino.getSndVal(), domino.getSndPos(), output);
    return output;
}
\end{lstlisting}

Um die Reihenfolge der Spieler am Ende eines Spiels berechnen zu k"onnen wird das Interface \emph{Comperable} implementiert \lstref{lst:player_compareTo}. Hierbei muss gew"ahrleistet sein, dass es sich bei dem "ubergebenen Objekt um eine Player-Referenz handelt, da sonst kein Vergleich durchgef"uhrt werden kann. Anschlie"send wird gepr"uft ob beide Spieler gleich viele Punkte vorzuweisen haben. Falls dies der Fall sein sollte wird die Gr"o"se der Distrikte verglichen. Zum Vergleichen der Distrikte wird die Methode \emph{getLargestDistrictSize} verwendet, welche nichts anderes tut als durch die Distriktliste des jeweiligen Spielers zu iterieren und zu z"ahlen wie viele Positionen die Liste jeweils h"alt. Am Ende wird die maximale Anzahl ausgegeben. 
Falls die Anzahl der Punkte nicht "ubereinstimmt werden einfach diese verglichen (beziehungsweise voneinander abgezogen).

Um das Ergebnis auf der Gui darstellen zu k"onnen wird ein TreeItem generiert. Dieses beinhaltet den Spielernamen, die erzielten Punkte und eine genaue Aufschl"usselung welche Punkte aus welchem Distrikt stammen. 
\begin{lstlisting}[style=CodeHighlighting,float,caption=Player - compareTo,label=lst:player_compareTo]
public int compareTo(Object o) {
    assert null != o && (o instanceof Player);
    Player other = (Player) o;
    if (other.getBoardPoints() == this.getBoardPoints()) {
        return this.getLargestDistrictSize() - other.getLargestDistrictSize();
    } else {
        return this.getBoardPoints() - other.getBoardPoints();
    }
}
\end{lstlisting}
 
