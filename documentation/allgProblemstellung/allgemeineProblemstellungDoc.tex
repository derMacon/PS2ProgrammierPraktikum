\part{Allgemeine Problemstellung}
Zu implementieren ist ein Dominospiel, bei dem vier Spieler jeweils ihre eigene Stadt gestalten. Ziel des Spiels ist es, möglichst viele Stadtteile mit Prestige zu gestalten.
\cite{aufgabenstellung}

\section{Spielregeln}
Jeder Spieler besitzt ein eigenes 5*5-Zellen großes Spielfeld und legt zu Beginn sein Stadtzentrum mittig ab.
Ein Spielbeutel für alle Spieler enthält 48 Spielkarten in der Größe von zwei Zellen, die auf ihren zwei Hälften jeweils einen (evtl. auch den gleichen) Stadtteiltyp anzeigen. Die Stadtteiltypen unterscheiden sich durch Bild und Hintergrundfarbe voneinander. Jede Spielkarte besitzt eine definierte Wertigkeit. Auf manchen Stadtteilen sind zusätzlich ein bis drei Prestigesymbole abgebildet.
Es werden vier Karten gezogen und im ersten Auswahlbereich angezeigt. Dabei wird die niederwertigste Karte zuoberst, die höchstwertigste zuunterst einsortiert. Der erste Spieler markiert die Karte im Auswahlbereich, die er gerne nehmen würde, die anderen Spieler treffen ihre Auswahl der Reihe nach ebenfalls und markieren die jeweils gewünschte Karte.
Wurden alle Karten markiert, dann werden wieder vier Karten gezogen und ebenso sortiert im zweiten Auswahlbereich angezeigt.


\section{Spielablauf}
Derjenige, der die oberste Karte im ersten Auswahlbereich markiert hat, beginnt eine Runde, es folgen der Reihe nach die Spieler, die die jeweils darunterliegende Karte markiert haben. In einer Runde wird zunächst eine Karte aus dem zweiten Auswahlbereich markiert und dadurch für die kommende Runde gewählt. Je wertvoller also seine markierte Karte in dieser Runde ist, desto später ist der Spieler am Zug und desto weniger Auswahl hat er für die kommende Runde.


\section{Anlegeregeln}
Die erste Karte muss an das Stadtzentrum angrenzen. An das Stadtzentrum darf jeder Stadtteil angrenzen. Legt man eine Karte an eine andere Karte an, so muss mindestens eine Hälfte mit einer Seite an einen identischen Stadtteiltyp einer liegenden Karte angrenzen.
Passt die abzulegende Karte weder an das Stadtzentrum noch an eine bereits ausliegende Karte, so wird sie verworfen.
Alle Spielkarten müssen in das 5*5-Feld passen, keine Hälfte darf hinausragen. Das Stadtzentrum muss aber nicht in der Mitte liegen, sondern kann im Spielverlauf verschoben werden, wodurch sich alle bereits gelegten Karten mit verschieben.
Eine abgelegte Karte kann nicht verschoben werden.


\section{Spielende}
Wurden alle Spielkarten aus dem Beutel gezogen und von den Auswahlbereichen auf die Spielfelder platziert bzw. verworfen, werden die Punkte ermittelt.
\begin{itemize}
	\item Jede Stadt besteht aus mehreren Stadtteilen. Ein Stadtteil setzt sich aus waagerecht und/oder senkrecht verbundenen Zellen desselben Stadtteiltyps zusammen. Das Stadtzentrum zählt zu keinem Stadtteil dazu.
	\item Die Punkte eines Stadtteils ergeben sich aus der Anzahl seiner Zellen multipliziert mit der Anzahl darin enthaltener Prestigesymbole.
	\item Innerhalb einer Stadt kann es mehrere voneinander getrennte Stadtteile desselben Typs geben. Jeder Stadtteil ist einzeln auszuwerten.
	\item Stadtteile ohne Prestigesymbole bringen keine Punkte.
\end{itemize}
Für die Auswertung wird für jeden Spieler die Summe der Punkte seiner Gebiete ermittelt. Gewonnen hat der Spieler mit den meisten Punkten. Bei einem Gleichstand gewinnt der Spieler mit dem größten einzelnen Gebiet. Besteht auch hier Gleichstand, so siegen beide Spieler gleichermaßen.


\section{KI}
Außer dem menschlichen Spieler, der im Spiel stets beginnt, existieren 3 computergesteuerte Spieler. Diese sollen einer sehr primitiven Logik folgen:
\begin{itemize}
	\item bei der Auswahl wird die Karte markiert, mit der bei Auslage im eigenen Feld aktuell am meisten Punkte erzielt werden könnten
	\item dafür wird für jede freie Karte der Auswahlbank auf jeder freien Position des Spielfeldes und in jeder Rotation der Punktgewinn ermittelt
	\item bei Punktgleichheit mehrerer Positionen wird darauf geachtet, dass keine leeren Einzelzellen erzeugt werden
bei der Ablage wird die so ermittelte Position genutzt
	\item die mögliche Verschiebung des Stadtzentrums wird nicht durchgeführt
\end{itemize}
Wer möchte, kann zusätzlich intelligentere KIs implementieren, die z.B. das Stadtzentrum verschieben, die Kartenwahl der kommenden Runde einbeziehen, verhindern, dass andere Spieler viele Punkte erhalten, oder Schlüsse aus den bereits abgelegten Karten ziehen.


\section{Oberfläche}
Existieren müssen folgende Elemente:
\begin{itemize}
	\item ein Spielfeld für den menschlichen Spieler
	\item ein erster Auswahlbereich für die aktuelle Runde
	\item ein zweiter Auswahlbereich für die kommende Runde
	\item das Legen einer Karte auf das Spielfeld per Drag and Drop, gültige Ablegepositionen werden beim DragOver sichtbar markiert
	\item eine Möglichkeit, um die Karte zu verwerfen. Das kann z.B. ein Button sein, der zum Verwerfen der aktuellen 	Karte betätigt wird, oder auf den die aktuelle Karte gezogen wird. Verworfene Karten müssen nicht angezeigt werden.
	\item die Spielfelder der drei KI-Spieler, so dass die dort abgelegten Karten jederzeit erkennbar sind.
\end{itemize}
Mögliche Lösungen für das Legen einer Karte:
\begin{itemize}
	\item Die aktuell zu legende Karte kann in einer Drehbox erscheinen, in der die Karte durch einfaches Anklicken gedreht werden kann. Hat sie die gewünschte Orientierung erreicht, so kann die Karte per Drag and Drop auf das eigene Spielfeld gelegt oder verworfen werden.
	\item Die aktuell gedragte Karte wird durch Tastendruck unter dem Mauscursor gedreht.
\end{itemize}
Die Reihenfolge der Spieler muss erkennbar sein. Es muss also zugeordnet werden können, welches der angezeigten Spielfelder zu welcher Kartenauswahl im Auswahlbereich gehört (z.B. durch gleiche Symbole oder farbliche Markierungen an beiden Stellen).
Das Stadtzentrum eines Spielfeldes muss zusammen mit allen bereits gelegten Karten verschoben werden können, entweder per Drag and Drop oder beispielsweise durch Buttons am Spielfeldrand. Dabei dürfen keine Stadtteile aus dem Spielfeld geschoben werden.
Die Auswertung eines Spiels muss für jeden Spieler die erreichten Punkte pro Stadtteiltyp darstellen.
Die Bedienung des Spiels muss intuitiv möglich sein für jemanden, der die Spielregeln kennt.
Die Größe des Fensters darf zu Spielbeginn höchstens 1600 * 900 Pixel betragen.

\section{Karten}
Die für ein Spiel vorhandenen Karten sind in dieser Datei definiert. Die zu den Stadtteiltypen gehörigen Bilder findet man hier.
Pro Zeile wird eine Karte mit ihren beiden Hälften und ihrem Wert festgelegt:
\verb|<Art><Symbolanzahl>,<Art><Symbolanzahl>,<Wert>|
Art ist dabei der Anfangsbuchstabe eines Stadtteiltyps (Amusement, Industry, Office, Park, Shopping, Home), die Symbolanzahl eine Ziffer von 0 bis 3. Eine mögliche Zeile wäre also \emph{H1,P0,24}
für eine Karte mit einem Symbol auf einem Haus und ohne Symbol in einem Park und einem Wert von 24 Punkten.

\section{Log}
In einer Datei (gleichzeitig auch auf System.out) sind durchgeführte Aktionen zu protokollieren. Der zuerst angegebene Stadtteil einer Karte ist dabei immer der an der angegebenen Position, bei horizontaler Ausrichtung liegt der zweite Stadtteil rechts davon, bei vertikaler darunter. Beispiel:\\
\\BOT1 chose [H1, P0] at index 1 for next round
\\BOT1 put [A0, P2] horizontally to (1, 2)\\
\\HUMAN chose [P0, S0] at index 0 for next round
\\HUMAN dragged center to (2, 3)
\\HUMAN put [A0, A0] vertically to (0, 0)\\
\\BOT3 chose [O0,I2] at index 3 for next round
\\BOT3 did not use [O0, A1]

\section{Spielstand}
Der aktuelle Spielstand soll gespeichert und geladen werden können. Laden/Speichern soll nur möglich sein, wenn der menschliche Spieler am Zug ist. Eine Spielstandsdatei enthält die 4 Spielfelder der Spieler in ihrer Reihenfolge (das erste Feld gehört immer dem menschlichen Spieler 0), die zwei Auswahlbereiche und die im Beutel verbliebenen Karten mit folgenden Bereichen:\\

\begin{addmargin}[50pt]{0pt}
<Spielfeld>
\\<Spielfeld>
\\<Spielfeld>
\\<Spielfeld>
\\<B"anke>
\\<Beutel>\\
\end{addmargin}

Die einzelnen Bereiche enthalten jeweils eine Einf"uhrungszeile (einen Kommentar) gefolgt von Inhaltsangaben:
\begin{itemize}
	\item Ein Spielfeld enthält einen Kommentar, zu wem es gehört, und in Folge für jede Zelle eine Inhaltsangabe:
	\begin{itemize}
		\item '--' für eine nicht belegte Zelle,
		\item 'CC' für das Stadtzentrum und
		\item zwei Buchstaben für eine H"alftenbeschreibung.
	\end{itemize}
	\item Die Zellen sind durch Leerzeichen separiert. Beispiel:
	\\ <Spielfeld>
	\\-- -- -- -- --
	\\-- -- H1 P0 --
	\\-- -- CC -- --
	\\-- -- -- -- --
	\\-- -- -- -- --
	\item Die Bänke enthalten Angaben für die aufliegenden Karten und von wem diese bereits gewählt wurde. Die erste Bank kann weniger als vier Karten enthalten, in entsprechender Anzahl enthält die zweite Bank dann bereits Markierungen (sonst '-' für fehlende Markierung). Die erste Bank wird als erste Markierung immer Spieler 0 (den menschlichen Spieler) aufweisen. Beispiel:
	\\<B"anke>
  \\0 H1P0,2 P0O1,3 I1P0
  \\- P0P0,- A0A0,1 H0A0,- P1H0
  \item Der Beutel enthält alle im Beutel befindlichen Karten kommasepariert. Im folgenden Beispiel befinden sich nur noch 4 Karten im Beutel
  \\<Beutel>
  \\P0P0,P0P0,A1H0,I2P0
\end{itemize}

